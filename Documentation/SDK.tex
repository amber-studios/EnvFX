\documentclass[a4paper]{article}
% \setlength{\textwidth}{15cm}
% \setlength{\textheight}{22cm}

\usepackage{alltt}
\usepackage{parskip}
\usepackage[top=3.5cm, bottom=3.5cm, left=3cm, right=3cm]{geometry}
\usepackage{graphicx}

\makeindex

\begin{document}

\title{ {\bf SharpE} \break \break
{\bf Service Development Kit }\break 
\normalsize www.sharpenviro.com}
\date{Version 0.7.3.0 (TD3) \break \break 20.08.2007}
\maketitle
\thispagestyle{empty}
\tableofcontents
\thispagestyle{empty}
\newpage
\setcounter{page}{1}


\section{Introduction}
  This document will explain how to create new Services for the SharpE Core Application.\\
  All example source code is related to the template service which is included in this package.\\
  The template service is designed to be compiled with the free Turbo Explorer version of Delphi.

  \medskip
  No third party components are required to compile the template service.
  
\section{The service project file}
\subsection{How does it work}
  A SharpCore service is a dynamic link library (dll) which is loaded by the SharpCore host application. Services are invisible 'applications' which are providing functionality to the shell by running in the background.
  
  The service dll is controlled by exporting a few pre defined functions which are then executed by SharpCore.

\subsection{File content and exported functions}
  The .dpr library file contains the exported functions of the service. Whenever SharpCore wants to do something with a service those functions are called. \\
  It is very important that all functions described below exist in a service dll file. If one of those functions is not exported then the service will not be loaded by SharpBar! 

  Services should be compiled with enabled vcl and rtl runtime packages. (It is possible to compile a service without runtime packages but this will increase the memory usage of the service) \\
  To enable compiling with runtime packages check {\it Project - Options - Packages} in Delphi and make sure that compile with runtime packages is checked and that the edit box below contains {\it vcl;rtl}

    The following list shows all the functions which have to be exported by a service.
  
  \medskip
  \begin{description}
    \item[]{\alltt {\bf function} Start(owner: hwnd): hwnd;}  \medbreak    
      The Start function is called when a service is started by SharpCore.
      
      In this function a service should initialize everything it needs to run in the background (hooks, actions, hotkeys, windows, ...)
      
      The {it owner: hwnd} parameter is the handle to the main window of the SharpCore host application
      
      The return value is not used at the moment, so it's not necessary to return any special handle.
      
      \bigbreak
    \item[]{\alltt {\bf procedure} Stop;}  \medbreak
      The Stop procedure is called when a service is stopped by SharpCore
      
      When this procedure is called a service must make sure that everything which was created in the Start procedure is properly given free. After calling the stop procedure a service is no longer allowed to have any existing windows, actions, hooks, classes, ... which might make SharpCore unable to unload the dll file.
      
      \bigbreak
      
    \item[]{\alltt {\bf function} SCMsg(msg: string): integer;}  \medbreak
      Using the SCMsg function is optional and not required. If a service exports this function then it will be called when a message is beeing send to this service.
      
      The {it msg : string} parameter specifies the message which was send to the service.
      
      Only services which are supposed to receive and handle messages which are send to service by other parts of the shell should export this function.
      
      \bigbreak
  \end{description}
  
\section{Global Windows Messages}
  A service normaly can't receive global window message broadcasts because a service isn't using a window. However for some services it might be important to react on global window messages like \verb+WM_DISPLAYCHANGE+ or even to SharpE wide shell broadcasts like the \verb+WM_SETTINGSCHANGE+ message.
  
  You can easily make a service receive global window message by creating an invisible 'dummy' window.
  
  First you have to define a new class which will hold the message handler. It's also necessary to declar a new var to store the window handle of the new message window.
  \begin{alltt}
01  type
02    TActionEvent = Class(TObject)
03      Procedure MessageHandler(var Message: TMessage);
04    end;
05
06  var
07    AE:TActionEvent;
08    h:THandle;
  \end{alltt}
  
  Now in the {\it Start} function of the service the class and the message window have to be created.
  \begin{alltt}
01  ae := TActionEvent.Create;
02  h := allocatehwnd(Ae.MessageHandler);
  \end{alltt}

  Of course both the class and the message window have to be given free in the In the {\it Stop} procedure of the service.
  \begin{alltt}
01  DeallocateHWnd(h);
02  AE.Free;
  \end{alltt}  
  
  The last step is it to declare the actual message handler of our TActionEvent class which will handle the global window messages.
  \begin{alltt}
01  procedure TActionEvent.MessageHandler(var Message: TMessage);
02  begin
03
04  end;
  \end{alltt}
  The {\it Message} parameter in this function can be used to handle the received message.
   
\end{document}